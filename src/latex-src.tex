\documentclass[a4paper,12pt]{article}
\usepackage[utf8]{inputenc}
\usepackage[spanish]{babel}
\begin{document}
\title{Títulodel artículo}
\author{Nombre y Apellido\\
        Técnicas Experimentales~\footnote{Universidad de La laguna}
        }
\date{\today}
\maketitle
\begin{abstract}
 En \LaTeX{}~\cite{Lam:86} es sencillo escribir expresiones
 matemáticas como $a=\sum-{i=1}^{10} {x-i}^{3}$
 y deben ser escritas entre dos símbolos \$.
 Los superíndices se obtienen con el símbolo \^{}, y 
 los subíndices con el símbolo \-.
 Por ejemplo: $x^2 \times y ^{\alpha + \beta}$.
 También se pueden escribir fórmulas centrasas:
 \[h[ 2=a^2 + b^2\]
 \end{abstract}
  
 \section{Primera Versión} 
 Si simplemente se desea escribir texto normal en LaTeX,
 sin complicadas f\'ormulas matem\'ticas o efectos especiales 
 como cambios de fuente, entonces simplemente tiene que escribir
 en espa\~nol normalmente.
 Si desea cambiar de párrafo ha de dejar una línea en blanco o bien
 utilizar el comando \par
 No es necesario preocuparse de la sangría automáticamente con la excepción
 Todos los párrafos se sangrarán automáticamente con la excepción
 del primer párrafo de una sección
 Se ha de distinguir entre la comilla simple 'izquierda'
 y la comilla simple 'derecha' cuando s eescribe en el ordenador.
 En el caso de que se quieran utilizar comillas dobles se han de 
 escribir dos caracteres 'comilla simple' seguidos, esto es, 
 "comillas dobles"
 También se ha de tener cuidado con los guiones: se utiliza un único
 guión para la separación de sílabas, mientras que se utilizan
 tres guiones seguidos para producir un guión de los que se usan 
 como signo de puntuación --- como en esta oración.
 \bigskip
 \begin{tabular}{|l|c|c|}
 \hline
   Nombre & Edad & Nota \\ \hline
   Pepe   &   24 &   10 \\ \hline
   Juan   &   19 &    8 \\ \hline
   Luis   &   21 &    9 \\ \hline
 \end {tabular}  
 \end{document}
