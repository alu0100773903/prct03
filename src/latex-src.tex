\documentclass[a4paper,10pt]{letter}
\begin{document}
Si simplemente se desea escribir texto normal en LaTeX,
 sin complicadas formulas matem ́aticas o efectos especiales
 como cambios de fuente, entonces simplemente tiene que escribir
 en espa~nol normalmente.
 Si desea cambiar de párrafo ha de dejar una línea en blanco o bien
 utilizar el comando \par.
 No es necesario preocuparse de la sangría de los párrafos:
 todos los parrafos se sangrar ́an autom ́aticamente con la excepción
 del primer párrafo de una sección.
 Se ha de distinguir entre la comilla simple ‘izquierda’
 y la comilla simple ‘derecha’ cuando se escribe en el ordenador.
 En el caso de que se quieran utilizar comillas dobles se han de
 escribir dos caracteres ‘comilla simple’ seguidos, esto es,
 ‘‘comillas dobles’’.
 También se ha de tener cuidado con los guiones: se utiliza un ́unico
 guión para la separaci ́on de s ́ılabas, mientras que se utilizan
 tres guiones seguidos para producir un gui ́on de los que se usan
 como signo de puntuación --- como en esta oración.
\end{document}